\documentclass[]{book}
\usepackage{lmodern}
\usepackage{amssymb,amsmath}
\usepackage{ifxetex,ifluatex}
\usepackage{fixltx2e} % provides \textsubscript
\ifnum 0\ifxetex 1\fi\ifluatex 1\fi=0 % if pdftex
  \usepackage[T1]{fontenc}
  \usepackage[utf8]{inputenc}
\else % if luatex or xelatex
  \ifxetex
    \usepackage{mathspec}
  \else
    \usepackage{fontspec}
  \fi
  \defaultfontfeatures{Ligatures=TeX,Scale=MatchLowercase}
\fi
% use upquote if available, for straight quotes in verbatim environments
\IfFileExists{upquote.sty}{\usepackage{upquote}}{}
% use microtype if available
\IfFileExists{microtype.sty}{%
\usepackage{microtype}
\UseMicrotypeSet[protrusion]{basicmath} % disable protrusion for tt fonts
}{}
\usepackage[margin=1in]{geometry}
\usepackage{hyperref}
\hypersetup{unicode=true,
            pdftitle={Applied Longitudinal Data Analysis},
            pdfauthor={Josh Jackson},
            pdfborder={0 0 0},
            breaklinks=true}
\urlstyle{same}  % don't use monospace font for urls
\usepackage{natbib}
\bibliographystyle{apalike}
\usepackage{color}
\usepackage{fancyvrb}
\newcommand{\VerbBar}{|}
\newcommand{\VERB}{\Verb[commandchars=\\\{\}]}
\DefineVerbatimEnvironment{Highlighting}{Verbatim}{commandchars=\\\{\}}
% Add ',fontsize=\small' for more characters per line
\usepackage{framed}
\definecolor{shadecolor}{RGB}{248,248,248}
\newenvironment{Shaded}{\begin{snugshade}}{\end{snugshade}}
\newcommand{\KeywordTok}[1]{\textcolor[rgb]{0.13,0.29,0.53}{\textbf{{#1}}}}
\newcommand{\DataTypeTok}[1]{\textcolor[rgb]{0.13,0.29,0.53}{{#1}}}
\newcommand{\DecValTok}[1]{\textcolor[rgb]{0.00,0.00,0.81}{{#1}}}
\newcommand{\BaseNTok}[1]{\textcolor[rgb]{0.00,0.00,0.81}{{#1}}}
\newcommand{\FloatTok}[1]{\textcolor[rgb]{0.00,0.00,0.81}{{#1}}}
\newcommand{\ConstantTok}[1]{\textcolor[rgb]{0.00,0.00,0.00}{{#1}}}
\newcommand{\CharTok}[1]{\textcolor[rgb]{0.31,0.60,0.02}{{#1}}}
\newcommand{\SpecialCharTok}[1]{\textcolor[rgb]{0.00,0.00,0.00}{{#1}}}
\newcommand{\StringTok}[1]{\textcolor[rgb]{0.31,0.60,0.02}{{#1}}}
\newcommand{\VerbatimStringTok}[1]{\textcolor[rgb]{0.31,0.60,0.02}{{#1}}}
\newcommand{\SpecialStringTok}[1]{\textcolor[rgb]{0.31,0.60,0.02}{{#1}}}
\newcommand{\ImportTok}[1]{{#1}}
\newcommand{\CommentTok}[1]{\textcolor[rgb]{0.56,0.35,0.01}{\textit{{#1}}}}
\newcommand{\DocumentationTok}[1]{\textcolor[rgb]{0.56,0.35,0.01}{\textbf{\textit{{#1}}}}}
\newcommand{\AnnotationTok}[1]{\textcolor[rgb]{0.56,0.35,0.01}{\textbf{\textit{{#1}}}}}
\newcommand{\CommentVarTok}[1]{\textcolor[rgb]{0.56,0.35,0.01}{\textbf{\textit{{#1}}}}}
\newcommand{\OtherTok}[1]{\textcolor[rgb]{0.56,0.35,0.01}{{#1}}}
\newcommand{\FunctionTok}[1]{\textcolor[rgb]{0.00,0.00,0.00}{{#1}}}
\newcommand{\VariableTok}[1]{\textcolor[rgb]{0.00,0.00,0.00}{{#1}}}
\newcommand{\ControlFlowTok}[1]{\textcolor[rgb]{0.13,0.29,0.53}{\textbf{{#1}}}}
\newcommand{\OperatorTok}[1]{\textcolor[rgb]{0.81,0.36,0.00}{\textbf{{#1}}}}
\newcommand{\BuiltInTok}[1]{{#1}}
\newcommand{\ExtensionTok}[1]{{#1}}
\newcommand{\PreprocessorTok}[1]{\textcolor[rgb]{0.56,0.35,0.01}{\textit{{#1}}}}
\newcommand{\AttributeTok}[1]{\textcolor[rgb]{0.77,0.63,0.00}{{#1}}}
\newcommand{\RegionMarkerTok}[1]{{#1}}
\newcommand{\InformationTok}[1]{\textcolor[rgb]{0.56,0.35,0.01}{\textbf{\textit{{#1}}}}}
\newcommand{\WarningTok}[1]{\textcolor[rgb]{0.56,0.35,0.01}{\textbf{\textit{{#1}}}}}
\newcommand{\AlertTok}[1]{\textcolor[rgb]{0.94,0.16,0.16}{{#1}}}
\newcommand{\ErrorTok}[1]{\textcolor[rgb]{0.64,0.00,0.00}{\textbf{{#1}}}}
\newcommand{\NormalTok}[1]{{#1}}
\usepackage{longtable,booktabs}
\usepackage{graphicx,grffile}
\makeatletter
\def\maxwidth{\ifdim\Gin@nat@width>\linewidth\linewidth\else\Gin@nat@width\fi}
\def\maxheight{\ifdim\Gin@nat@height>\textheight\textheight\else\Gin@nat@height\fi}
\makeatother
% Scale images if necessary, so that they will not overflow the page
% margins by default, and it is still possible to overwrite the defaults
% using explicit options in \includegraphics[width, height, ...]{}
\setkeys{Gin}{width=\maxwidth,height=\maxheight,keepaspectratio}
\IfFileExists{parskip.sty}{%
\usepackage{parskip}
}{% else
\setlength{\parindent}{0pt}
\setlength{\parskip}{6pt plus 2pt minus 1pt}
}
\setlength{\emergencystretch}{3em}  % prevent overfull lines
\providecommand{\tightlist}{%
  \setlength{\itemsep}{0pt}\setlength{\parskip}{0pt}}
\setcounter{secnumdepth}{5}
% Redefines (sub)paragraphs to behave more like sections
\ifx\paragraph\undefined\else
\let\oldparagraph\paragraph
\renewcommand{\paragraph}[1]{\oldparagraph{#1}\mbox{}}
\fi
\ifx\subparagraph\undefined\else
\let\oldsubparagraph\subparagraph
\renewcommand{\subparagraph}[1]{\oldsubparagraph{#1}\mbox{}}
\fi

%%% Use protect on footnotes to avoid problems with footnotes in titles
\let\rmarkdownfootnote\footnote%
\def\footnote{\protect\rmarkdownfootnote}

%%% Change title format to be more compact
\usepackage{titling}

% Create subtitle command for use in maketitle
\newcommand{\subtitle}[1]{
  \posttitle{
    \begin{center}\large#1\end{center}
    }
}

\setlength{\droptitle}{-2em}
  \title{Applied Longitudinal Data Analysis}
  \pretitle{\vspace{\droptitle}\centering\huge}
  \posttitle{\par}
  \author{Josh Jackson}
  \preauthor{\centering\large\emph}
  \postauthor{\par}
  \predate{\centering\large\emph}
  \postdate{\par}
  \date{2017-07-12}

\usepackage{booktabs}
\usepackage{amsthm}
\makeatletter
\def\thm@space@setup{%
  \thm@preskip=8pt plus 2pt minus 4pt
  \thm@postskip=\thm@preskip
}
\makeatother

\begin{document}
\maketitle

{
\setcounter{tocdepth}{1}
\tableofcontents
}
\chapter*{}\label{section}
\addcontentsline{toc}{chapter}{}

Using R to test development questions using MLM and SEM

\chapter{Syllabus}\label{syllabus}

\textbf{Course Descrition}

\textbf{Class textbook}

\textbf{Grading}

\textbf{Schedule}

\begin{longtable}[]{@{}llll@{}}
\toprule
Week & Date & Topic & Readings\tabularnewline
\midrule
\endhead
1 & 1/17 & Motivation, terms, concepts and graphing &\tabularnewline
2 & 1/24 & Growth curves; MLM in R: packages and procedures
&\tabularnewline
3 & 2/2 & Conditional (Leve 1 and 2 predictors) MLM models
&\tabularnewline
4 & 2/7 & Polynomial, piecewise and spline models &\tabularnewline
5 & 2/14 & Intensive data anlysis/within person fluctuations p1
&\tabularnewline
6 & 2/21 & Intensive data anlysis/within person fluctuations p2
&\tabularnewline
7 & 2/28 & Experimental approaches and 2 wave data &\tabularnewline
8 & 3/7 & Dyadic Models &\tabularnewline
9 & 3/14 & \textbf{Spring Break!} &\tabularnewline
10 & 3/21 & SEM and Latent Grown (curve) Models &\tabularnewline
11 & 3/28 & Scaling, MI, and Second order Model &\tabularnewline
12 & 4/4 & Multiple group models &\tabularnewline
13 & 4/11 & Flexible SEM models (LCM, STATE-TRAIT; ALT-SR)
&\tabularnewline
14 & 4/18 & Mixture Models &\tabularnewline
15 & 4/25 & Biometric or Survival or ? or catch up &\tabularnewline
\bottomrule
\end{longtable}

\chapter{LDA basics}\label{lda-basics}

\section{Motivation, terms, concepts}\label{motivation-terms-concepts}

\subsection{Why longtiduinal?}\label{why-longtiduinal}

At least 5 reasons:

\begin{enumerate}
\def\labelenumi{\arabic{enumi}.}
\item
  Identification of intraindiviaul change (and stability). Do you
  increase or decrease with time or age. Is this pattern monotonic?
  Should this best be conceptualized as a stable process or something
  that is more dynamic? On average how do people change?
\item
  Inter-individual differences in intraindividual change. Does everyone
  change the same? Do some people start higher but change less? Do some
  increase while some decrease?
\item
  Examine joint relationship among intraindividual change for two or
  more constructs. If variable X goes up does variable Y also go up
  across time? Does this always happen or only during certain times? Is
  this assocaition due to a third variable or does it mean that change
  occurs for similar reasons?
\item
  Determinants of intraindividual change. What are the repeated
  experiences that can push construct X around. Do these have similar
  effects at all times?
\item
  Determinants of interindividual differences in intraindividual change.
  Do events, background characteristics, interventions or other between
  person characteristic shape why certain people change while others
  don't?
\end{enumerate}

\subsection{Types of change}\label{types-of-change}

There are many ways to think of change and stability. We will only have
time to go into a few of these types, but it is helpful to think about
what type you are interested in when you plan a project or sit down to
analyze data.

\begin{enumerate}
\def\labelenumi{\arabic{enumi}.}
\item
  Differential / rank order consitency/ rank order stability
\item
  Mean level/ absolute change
\item
  Individual differences in change
\item
  Ipsative Change
\item
  Heterotypic change
\end{enumerate}

\subsection{Between person versus within
person}\label{between-person-versus-within-person}

Or in other words these are the shortened version of interindividaul
differences in change versus intraindividaul differences. Refers to
across people versus within a particular person. Often we are interested
in both simultaneously.

Related to Fixed effects and Random effects.

\subsection{Trajectories, curves, change, growth\ldots{} oh
my}\label{trajectories-curves-change-growth-oh-my}

How do we refer to `change'? Usually it is easier to refer to
pictorially or in terms of an equation. Putting a word onto it usually
causes some confusion, which is why there are a lot of redundent terms
in the literature. All of these might refer to the same thing when used
within a model. However, the names of some models use these terms
differently and thus can refer to different models or conditions that
you are working with. In this class I will try to point out the
important differences but you will be fine if you supplement your terms
with graphs or equations.

\section{Data structures and data
analysis}\label{data-structures-and-data-analysis}

\subsection{MLM \& SEM}\label{mlm-sem}

In this class (and in the field) two primary techniques are used with
longitudinal models: MLM and SEM. At some levels they are completely
equivalent. At others, one is better than the other and vice versa.

MLM/HLM is a simple extension of regression. As a result it is easy to
interpret and impliment. In terms of longituduinal data it is easier to
run models when the time of measurement differs from person to person.
For this class we will use lme4 as our MLM program but there are many
others we could use e.g., nlme.

SEM is related to regression in that regression is a subset of SEM
techniques. In other words, an SEM program could run a simple regression
analysis.

\subsection{Wide and Long form}\label{wide-and-long-form}

\begin{Shaded}
\begin{Highlighting}[]
\KeywordTok{library}\NormalTok{(tidyr)}
\end{Highlighting}
\end{Shaded}

\section{Graphing}\label{graphing}

\begin{Shaded}
\begin{Highlighting}[]
\KeywordTok{library}\NormalTok{(ggplot2)}
\end{Highlighting}
\end{Shaded}

\bibliography{packages.bib,book.bib}


\end{document}
